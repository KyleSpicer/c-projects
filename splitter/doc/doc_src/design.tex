\usepackage[legalpaper, portrait, margin=1in]{geometry}
\documentclass[12pt]{article}
\usepackage{layout}

\begin{document}
\title{Splitter Design}
\author{\Kyle Spicer\\ 170D WOBC \#22-02}
\date{October 6th, 2022}

\maketitle

\section{Project Summary:}\label{Project Summary}
A musician is testing sequences of notes to see which phrases have good splits. A \textit{good split} of a sequence of notes is when the sequence can be split into two sub-strings, each with the same number of distinct letters. 
\newline\newline
For instance, 'ccbbab' has two possible good splits: "ccb, bab", "ccbb, ab"


\section{Architecture:}\label{Architecture}
Experiment and find the best way to read in strings, validate, and store the information for further use. Determine what the most efficient way is to store and utilize data.
\subsection{Data}
1. Make file with appropriate flags / targets\newline
2. splitter.c splitter.h splitter-funcs.c\newline
3. design doc, writeup, test plan (must be PDFs)\newline

\subsection{Constants}
1. MAX\_LINE\_LENGTH 128

\subsection{Significant Functions}
1. read in the file \newline
2. validate input \newline
3. parse strings \newline
4. logic for algorithm \newline
5. displaying results\newline
6. close file

\newline


\section{Program Flow:}\label{Program Flow}
1. open file, if unable to open file prompt for user input\newline
2. read in a single line, if line isn't ASCII letter, publish error message\newline
3. if line is valid, calculate how many \textit{good splits} are possible and print\newline
4. continue reading/printing until EOF is reached\newline
5. close file prior to exiting program\newline
\newline


\end{document}


